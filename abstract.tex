This report provides a comprehensive analysis of the YOLO (You Only Look Once) computer vision model, a state-of-the-art object detection algorithm that has significantly impacted various fields. The report traces the evolution of YOLO from its initial version to the latest iterations, highlighting the key architectural advancements and performance improvements that have occurred over time. It deep dive into the core components of YOLO, including the backbone, neck, and head, as well as the fundamental concepts underpinning its operation, such as grid cells, bounding box prediction, confidence scores, and class probabilities. The crucial role of the loss function and the Non-Maximum Suppression technique in refining YOLO's predictions is also examined.
The report further explores the diverse applications and substantial impact of YOLO in biomedical science, here goes blh blh blh proposal for biomed Similarly, the application of YOLO in agriculture is investigated, here goes blah blah for proposal 
The inherent strengths of YOLO, including its speed, accuracy, and versatility, are discussed alongside its limitations, such as challenges in detecting small or overlapping objects and its computational demands. Finally, the report proposes a novel application of YOLO in biomedical science: same here goes the proposal blah blah blah It concludes by outlining potential future research directions aimed at further enhancing YOLO's capabilities and addressing its limitations in both biomedical science and other industries. Appendices containing illustrative Python code snippets and a comparative table of YOLO versions' performance supplement the main body of the report.
