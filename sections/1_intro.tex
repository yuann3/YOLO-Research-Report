\begin{multicols}{2}
\section{Introduction}
Object detection represents a fundamental task in computer vision, with transformative applications spanning self-driving vehicles, security infrastructure, and medical diagnostics. The ability to accurately identify and localize objects within images has revolutionized numerous fields by enabling novel automation capabilities \citep{8627998}. Among the diversity of object detection architectures developed in recent years, YOLO (You Only Look Once) has emerged as particularly significant due to its exceptional processing efficiency.

In contrast to traditional two-stage methodologies such as R-CNN and Fast R-CNN—which first identify potential regions of interest before classifying them \citep{7410526}—YOLO reformulates object detection as a unified regression problem that simultaneously determines bounding box coordinates and object classifications in a single computational pass. This innovative approach, initially proposed by \citet{redmon2016lookonceunifiedrealtime}, dramatically transformed the speed-accuracy landscape in computer vision, enabling real-time processing without compromising detection reliability. Subsequent iterations including YOLO9000 \citep{8100173}, YOLOv3 \citep{redmon2018yolov3}, and YOLOv4 \citep{bochkovskiy2020yolov4} have progressively refined this architecture, yielding substantial performance improvements.

The unified detection paradigm introduced by YOLO has profound implications across diverse application domains. In biomedical contexts, where precise cellular identification and classification are critical for diagnostic assessments and research protocols, YOLO offers the potential to substantially reduce manual counting burdens while maintaining accuracy \citep{Chan2020}. Similarly, in agricultural settings, the accurate assessment of fruit ripeness enables optimal harvest timing, minimizes produce wastage, and enhances food quality \citep{Koirala2019}. Both applications benefit from YOLO's capacity for rapid, reliable object detection, though each presents unique implementation challenges.

Our research examines YOLO's efficacy specifically in these two domains: cellular detection in biomedical imaging and fruit ripeness determination in agriculture. These application areas present distinctive technical challenges—biomedical applications require exceptional precision when dealing with small, often visually similar objects \citep{electronics8030292}, while agricultural implementations must function reliably under variable environmental conditions with complex backgrounds and lighting variations \citep{KAMILARIS201870}.

Through systematic investigation, our work addresses several critical research questions: First, we examine how YOLO's architectural evolution through successive versions has enhanced its suitability for biomedical and agricultural applications. Second, we quantitatively assess YOLO's performance in cellular detection and fruit ripeness assessment relative to alternative methodologies. Third, we identify specific modifications that optimize YOLO's effectiveness for these specialized domains. Finally, we delineate remaining challenges that must be resolved for successful real-world deployment.

By conducting comprehensive evaluations and comparative analyses, our research illuminates YOLO's strengths and limitations in biomedical and agricultural contexts, with significant implications for both computer vision research and practical implementation in these fields. Our findings provide valuable insights for researchers advancing object detection methodologies as well as practitioners seeking to deploy these systems in medical and agricultural environments.
\end{multicols}